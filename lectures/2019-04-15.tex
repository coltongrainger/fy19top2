\begin{quote}
    \textit{``Does the torus $S^1 \times S^1$ have a cohomology ring%
        \footnote{\textsf{Hint.} The cohomology ring for $S^1$ is to $\RP^2$ as $\bl$ is to $\bl$ (?) with spheres adjoined (!?). 
        Consider the homogeneous elements of the ring. 
        Just describe the ring. 
        For example, when algebraic topologists write $\Z[x]/\ang{x^4}$, it is understood to be \emph{a homogeneous ring}. Addition is \emph{strictly levelwise}. Products are defined between two homogeneous elements of non-homogeneous degrees.}
    that's not a polynomial ring? Does any space?''}
\end{quote}

Recall that last week, we proved the Eilenberg-Zilber chain map $\theta$ was a natural homotopy inverse of the cross product~$\times$. 
And last lecture, knowing $\theta$ and $\times$ are (chain-)homotopy equivalences, we exploited the chain equivalence $\theta \colon {\Delta}_{*}\paren{X \times Y} \to {\Delta}_{*}\paren{X} \otimes {\Delta}_{*}\paren{Y}$
to define the cross product and the cup product on cohomology groups. 
\begin{note}[]
   It's not at all apparent, however, what map $\times$ induces on cohomology:
   \begin{equation}
       \label{crossinduced}
       \times \colon H^*\paren{ X; \Lambda } \otimes \coho{Y}{\Lambda} \overset{?}{\to} \coho{X\times Y}{\Lambda}.
   \end{equation}
    In general, there's no Künneth theorem for the cohomology cross product. 
    But, if the ring of coefficients $\Lambda$ is a field and either ${H}_{*}\paren{X;\Lambda}$ or ${H}_{*}\paren{Y;\Lambda}$ is of finite type, then the map in \eqref{crossinduced} is an isomorphism. 
    (E.g., if $\Lambda = \R$ and $X$ is a compact $n$-manifold.)
\end{note}
This lecture we'll give details for making computations with the cup product. Let $X$ and $Y$ be topological spaces and $\Lambda$ a commutative unital ring. 

\begin{todo}[]
    Say $f\colon A_* \to \Lambda$ and $g\colon B_* \to \Lambda$ are cochains in the complexes $A^*(\Lambda)$ and $B^*(\Lambda)$. 
    Check that the ring structure of $\Lambda \otimes \Lambda = \Lambda$ forces the definition
    \begin{align*}
        f \otimes g \colon  A_* \otimes B_* & \to \Lambda\\
        (f \otimes g)(a \otimes b) &= (-1)^{\deg a \deg g} f(a) g(b),
    \end{align*}
    for chains $a \in A_*$ and $b \in B_*$.
\end{todo}

\begin{coro}
    For the singular chain complexes over a topological space $X$, let $f, g$ be cochains in $\Delta^*(X)$ and $\alpha, \beta$ chains in $\Delta_*(X)$. 
    Then, $(f\otimes g)(\alpha \times \beta) = (-1)^{\deg \alpha \deg g} f(\alpha)g(\beta)$.
\end{coro}

Now, we certainly have an \term{evaluation} $\ev$ from the group of $p$-cochains $f \colon \Delta_p(X) \to \Lambda$ tensored with the group $\Delta_p(X)$ of $p$-chains,
\[
    \Delta^p(X; \Lambda) \otimes \Delta_p(X) \xrightarrow{\ev} \Lambda,
\]
defined by
\[
    \ev\colon f \otimes c \mapsto f(c).
\] 
In fact, $\ev$ induces a map on cohomology, which is denoted by the angle brackets
\[
    \ang{\bkt{f}, \bkt{c}} \in \Lambda.
\]

\begin{lem}[Krönecker pairing]
   The evaluation $\Delta^p(X; \Lambda) \otimes \Delta_p(X) \xrightarrow{\ev} \Lambda$ induces a $\Lambda$-linear map
   \begin{equation}
       \label{kpairing}
       H^p(X;\Lambda) \otimes H_p(X) \to \Lambda
   \end{equation}
   such that $\ang{\abs{f}, \abs{ c }}\mapsto f(c)$.
\end{lem}

\begin{proof}
    (\TODO: revise) The \term{Kronecker pairing} is the argument that $f(c)$ does not depend on representatives $f$ or $c$ (from the cochain, resp, chain complexes). Consider that in the proof of the universal coefficient theorem, we found a map $\beta$ from $\coho{X}{\Lambda}$ to $\Hom(H_p(X), \Lambda)$ such that $\abs{f} \mapsto \set{ \abs{c} \mapsto f(c) }$ gave a group homomorphism. Use this.
\end{proof}

\begin{rem}
    The \term{cap product} operation over a topological space $X$ is the above~\eqref{kpairing} pairing, which is ``given by combining the \term{Kronecker pairing} of the cohomology class with the image of the homology class under diagonal and using the Eilenberg-Zilber theorem.'' (See \url{https://ncatlab.org/nlab/show/cap+product}.)
\end{rem}

\begin{defn}[The cup product on cochains]
   \label{cupproductoncochains}
    Let $X \in \Top$. The diagonal map $d \colon X \to X \times X$, induces $d_\Delta \colon \Delta_*(X) \to \Delta_*(X)$. 
    Define a natural \term{diagonal approximation} $\Delta = d_\Delta \circ \theta$, where $\theta$ is the chain equivalence from the Eilenberg-Zilber theorem.
    \begin{equation*}
    \begin{gathered}\xymatrix@=1em{%
        % should this be injective?  <ccg, 2019-04-15> %
         % duh! free groups! 
        0 \ar[r]
        & \Delta_*(X) \ar@{->}[r]^-{d_\Delta} \ar@/_2pc/[rr]^\Delta
        & \Delta_*(X \times X) \ar[r]^-\theta 
        & \Delta_*(X) \otimes \Delta_*(X)
    }
    \end{gathered}
    \end{equation*}
    The \term{cup product of homogeneous cochains} $f$ and $g$ is
    \[
        f \smile g = (f \otimes g) \theta d_\Delta.
    \]
\end{defn}

\begin{note}[]
    The equation
    \[
        \delta(f \smile g) = \delta f \smile g + (-1)^{\deg}f \smile \delta g
    \]
    follows from the boundary formula for the cross product $\times$.
\end{note}

\begin{comp}[Properties of the cup product]{prop}{enumerate}
    \item The cup product is natural for $X$ in $\Top$ and $\Lambda$ in $\Ring$. Given a continuous map $\phi \colon X \to Y$ in $\Top$, the induced map on cohomology satisfies 
       \[
       \phi^*(\alpha \smile \beta) = \phi^*(\alpha) \smile \phi^*(\beta)
       \]
       for all homogeneous cochains $\alpha, \beta$ in $\Delta^*(X)$.
    \item $\alpha \smile 1 = \alpha = 1 \smile \alpha$, where $1$ is the class of the augmentation $\epsilon$ (\TODO. Be specific.)
    \item The cup product $\smile$ is associative.
    \item The cup product is \term{skew-commutative}:
       \[\alpha \smile \beta = (-1)^{\deg \alpha \deg \beta} \paren{ \beta \smile \alpha }. \] 
\end{comp}

\begin{defn}[Alexander--Whitney diagonal approximation]
   Let $\sigma \colon \Delta_n \to X$ be a singular $n$-simplex in $X$. The \term{Alexander--Whitney diagonal approximation} explicitly computes the image of $\sigma$ under the chain map $\Delta \colon \Delta_*(X) \to \Delta_*(X) \otimes \Delta_*(X)$ from the \term{front and back faces} of $\sigma$.
   \[
       \Delta \sigma  = \sum_{p + q = n} \norm{ \sigma }_\text{front}^p \otimes \norm{ \sigma }_\text{back}^q
   .\]
\end{defn}

\begin{todo}
    Any two chain maps $\Phi, \Psi \colon \Delta_*(X) \to \Delta_*(X\otimes X)$ that agree with the diagonal approximation
    \[
        \Delta(x) = x \otimes x \qq{in the $0$th degree}
    \]
    are chain homotopic: $\Phi \simeq \Psi$.
\end{todo}

\begin{prop}[Computing the cup product]
    Say $f$ and $g$ are in the cochains with degrees $p$ and $q$ respectively, such that $p+q = n$. Then
    \begin{align*}
        (f \smile g)(\sigma) &= (f \otimes g)(\Delta\sigma)\\
                           &= (f \otimes g) \paren{\sum_{i+j = n} \norm{ \sigma }_\text{front}^i \otimes \norm{ \sigma }_\text{back}^j} \\
                           &= (f \otimes g) \paren{ \norm{ \sigma }_\text{front}^p \otimes \norm{ \sigma }_\text{back}^q}  \\
                           & = (-1)^{\deg g \deg f} f(  \norm{ \sigma }_\text{front}^p ) g(\norm{ \sigma }_\text{back}^q) \quad (\text{an element of } \Lambda).
    \end{align*}
\end{prop}

\begin{todo}[A derivation from the cup product]
Let $A, B  \subset X$ in $\Top$ be open in $X$. Verify the following:

\begin{enumerate}
   \item $\Delta_*(A) + \Delta_*(B) \surj \Delta_*(A \smile B)$.
   \item $\coho{X, A}{\Lambda} \otimes \coho{X, B}{\Lambda} \to \coho{X, A}{\Lambda}$. \label{relativecoho}.
    
   \item From the snake lemma and~\eqref{relativecoho}, there's a long exact sequence
        \[
           \xymatrix{\cdots & \ar[l] H^{*+1}(X, A; \Lambda) & \ar[l]^-{\delta^{*}} H^*(A; \Lambda) & \ar[l]^-{i^*} H^*(X;\Lambda) & \ar[l] \cdots}
        \]
        that's natural in $X, A, B$ and $\Lambda$.
        \item The connecting homomorphism $\delta$ is a derivation (\TODO\ of what?) defined by 
% this is wrong:  <ccg, 2019-04-15> %
        \[
        \delta^*(\alpha \smile i^*(\beta)) = \alpha \smile \delta^*(\beta)
        .\]
\end{enumerate}
\end{todo}
