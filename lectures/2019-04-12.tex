\textsc{Announced!} In the fall semester, we'll run a reading course on an introduction to chromatic homotopy theory. We'll start from basics---the "homotopy category" and fibrations.

Today $\Lambda$ is a commutative unital ring. 

\begin{comp}{claim}[Properties]{enumerate}
    \item We have a sign convention for the maps of cohomological\footnote{or, dually, the homological degree of $\Hom(A_*, B_*)_{-p}$.} degree $p$
    \[\Hom(A_*, B_*)^p = \{f_i \colon A_i \to B_{i-p}\}.\]
    That is, if $f$ has cohomological degree $p$, then the chain map looks like
    \[\xymatrix@=0.7em{
        A_{i+1} \ar[r] \ar[d] & B_{i+1-p}\\
        A_{i} \ar[r] \ar[d] & B_{i-p}\\
        A_{i-1} \ar[r]  & B_{i-1-p}}.\]
    \item $\Hom(A_*, B_*)^p$ is a chain complex with differential 
    \[(\delta f) (a) = \partial (f(a)) - (-1)^{\deg f} f(\partial (a)).\]
    \item Consider the (chain map) element $f \in \Hom(A_*, B_*)$. Then $f$ is a cocycle if and only if $f$ has cohomological degree $0$.
    \item Consider the chain complex concentrated at degree $0$.
    \[G_* = \{0 \qq{ not in degree $0$ else $G$}\}.\]
    With our sign conventions, the degree of each (chain map) element $f \in  \in \Hom(A_p, G)$ is $(-1)^{\deg f + 1}$. 
    Therefore $\Hom(A_*, G)$  is a chain complex with differential $\delta f = (-1)^{\deg f + 1} f \partial$.
\end{comp}

\begin{note}[]
    Keep in mind that the resolution $F_*(G) \to G$ (for $G$ a finitely generated abelian group) induces a chain map.
    $$\xymatrix{
        0 \ar[r] \ar[d] & F_1 \ar[r] \ar[d]  & F_0 \ar[r] \ar[d] & 0 \ar[d] \\
        0  \ar[r] &  0 \ar[r] &  G \ar[r] & 0 }.$$
\end{note}

To handle products (especially products of CW-complexes) in singular homology, we proved there exists a natural bilinear map $\times$ (the cross product) from the product of two singular chain complexes to the chain complex of the product of two spaces. 
With $\times$, computed the boundary map of the chain complex ${\Delta}_{*}\paren{X \times Y}$ in  ${\Delta}_{*}\paren{X} \times {\Delta}_{*}\paren{Y}$, incidence degree. 
Taking the interval $I = Y$ or an interval $I$ gave us a proof of the homotopy axiom.Now we would like to compose the definitionsincidence degrees
We computed the homology of 
\begin{defn}[]
    $\times$ product extend the cross product
        $$\times \colon \Delta^p (X; \Lambda) \otimes \Delta^q (Y; \Lambda) \to \Delta^{p+q}(X\times Y; \Lambda)$$ 
    for (the cochains) $f, g \colon \Delta_p (X) \to \Lambda$ 
        $$f \otimes g \colon \Delta_p(X) \otimes \Delta_q(Y) \to \Lambda \otimes \Lambda.$$
    But we can also map 
        $$\Lambda \otimes \Lambda \xrightarrow{m} \Lambda$$ 
    so that $\sigma \otimes \tau \mapsto f(\sigma) \otimes g(\tau) \mapsto f(\sigma) g(\tau)$.
\end{defn}

Recall from the Eilenberg-Zilber theorem (which is only hard for singular chain complexes generated on spaces), there's a map $\Theta \colon \Delta_*(X \times Y) \to \Delta_*(X) \otimes \Delta_*(Y)$ for which we have the chain homotopies $\Theta \circ \times \sim \id$ and $\times \circ \theta \sim \id$. 
Define
    $$f \times g = f\otimes g \circ \Theta.$$

\begin{rem}[]
    Yu asked how $$\Theta \colon \Delta_p(X\times Y; \Lambda) \to (\Delta_*(x) \otimes \Delta_*(Y))_p$$ could be well defined.
    
    Fix $f \colon \Delta_p(X) \to \Lambda$, then extend to all of $\Delta_*(X)$ by varying $p$. \TODO.
\end{rem}

\begin{lem}[]
    $\delta (f \times g)  = \delta f \times g + (-1)^{\deg f} f \times \delta g$.
\end{lem}

\begin{proof}
\TODO\ (acyclic models).
\end{proof}

\begin{fact}[]
There's a natural (linear map out of the) product of homology groups
     $$\times \colon H^p(X; \Lambda) \otimes H^q(Y; \Lambda) 
         \to H^{p+q}(X \times Y; \Lambda)$$
such that 
     $$f \otimes g \mapsto f \times g,$$
which is induced by
$$\xymatrix{
    \Delta_*(X \times Y) \ar[r]^\Theta &
        \Delta_*(X) \otimes \Delta_*(Y) \ar[r]^{f\otimes g} &
        \Lambda
}.$$
\end{fact}

\begin{proof}
Let the unit $1 \in H^0(X; \ZZ)$ be the class of the augmentation $\epsilon \colon \Delta_0(X) \to\ZZ$.

The universal property of the product space $X \times Y$ is expressed by the projections
$$\xymatrix@=0.7em{
    X\times Y \ar[dr]^{\pi_Y} \ar[r]^{\pi_X} & X \\
    & Y 
}$$
which induce cochain maps 
$$\xymatrix@=0.7em{
   {\Delta}_{*}\paren{X \times Y} & {\Delta}^{*}\paren{X}  \ar[l]^{\pi^*_X} \\
                                  & {\Delta}^{*}\paren{Y}  \ar[ul]^{\pi^*_Y}
}$$
I claim that these cochain maps descend to cohomology. 
Explicitly, there' a homomorphism $H^*(X) \to H^*(X \times Y)$ pulling back $\alpha$ over $\pi^*_X$ to $\alpha \times 1$. By symmetric argument, $\beta \mapsto 1 \times \beta$.

Consider the "slice map" (\TODO\ Are these over categories?)
$$\xymatrix@=0.7em{
    \Delta_p (X) \otimes \Delta_0(X) \ar[r]^{\times} &
        \Delta_p(X \times Y) \ar[r]^{{\pi_X}_\Delta} &
        \Delta_p(X) \\
    \tau \otimes y \ar@{|->}[r]                      &
    (\tau \times y)(x) = (\tau(z), y) \ar@{|->} [r]  &
    \tau
}$$

\TODO\ Once extended linearly, we have a map on chains, therefore an induced map.
\end{proof}

\begin{ex}[]
\label{twist}
    How can we compare $\alpha \times \beta \in H^*(X\times Y; \Lambda)$ to $\beta \times \alpha \in H^*(Y\times X; \Lambda)$? There's a twist map $T \colon X \times Y \to Y \times X$ taking $(\alpha, \beta) \mapsto (\beta, \alpha)$, which induces a map on cohomology 
    \begin{equation*}
        \xymatrix{H^*(X\times Y; \Lambda) & \ar[l]^{T^*} H^*(Y\times X; \Lambda)}.
    \end{equation*}
\end{ex}

\begin{todo}[]
From example \ref{twist}, prove that the cup product is a ``graded''-commutative operation.
\end{todo}

To define the cup product, we need the diagonal map $d \colon X \to X \times X$ such that $d(x) = (x,x)$. The induced map $d^*$ back on cohomology should be pre-composed with the map out of the tensor product. The later map (out the tensor product) exists by application of the Eilenberg-Zilber theorem.
$$\xymatrix{
    H^*(X; \Lambda) \otimes H^*(X; \Lambda)  \ar[r]^{\times} & 
        H^*(X\times X; \Lambda) \ar[r]^{d^*} & 
        H^*(X; \Lambda)\\
    \alpha \otimes \beta \ar@{|->}[rr] & &
        \alpha\text{"cup"}\beta}.$$ 

Define $1 \in H^0(X; \ZZ)$ to be the class of $\epsilon \colon \Delta_0(X) \to\ZZ$.

From the product
$$\xymatrix{
    X\times Y \ar[dr]^{\pi_Y} \ar[r]^{\pi_X} & 
        X \\
    & Y 
}$$
we have the induced map on $H^*(X) \to H^*(X \times Y)$ taking $\alpha$ under $\pi^*_X$ to $\alpha \times 1$ (similarly $\beta \mapsto 1 \times \beta$).

In particular, consider the "slice map"
$$\xymatrix{
    \Delta_p (X) \otimes \Delta_0(X) \ar[r]^{\times} & 
        \Delta_p(X \times Y) \ar[r]^{{\pi_X}_\Delta} & 
        \Delta_p(X) \\
    \tau \otimes y \ar@{|->}[r] &
    (\tau \times y)(x) = (\tau(z), y) \ar@{|->} [r]& 
    \tau
}$$

\TODO\ Once extended linearly, we have a map on chains, therefore an induced map.

How can we compare $\alpha \times \beta \in H^*(X\times Y; \Lambda)$ to $\beta \times \alpha \in H^*(Y\times X; \Lambda)$?
There's a twist map $T \colon X \times Y \to Y \times X$ taking $(\alpha, \beta) \mapsto (\beta, \alpha)$,
which induces a map on cohomology
$$\xymatrix{H^*(X\times Y; \Lambda) & \ar[l]^{T^*} H^*(Y\times X; \Lambda)}.$$
*Note.* From this, one shows that the cup product is anti-commutative.

To finally define the cup product, we need the diagonal map $d \colon X \to X \times X$ such that $d(x) = (x,x)$. The induced map $d^*$ back on cohomology along with the map from the tensor product into the product (given by Eilenberg-Zilber) allow one to string together (this magic of topological spaces):
$$\xymatrix{
    H^*(X; \Lambda) \otimes H^*(X; \Lambda)  \ar[r]^{\times} & 
        H^*(X\times X; \Lambda) \ar[r]^{d^*} & 
        H^*(X; \Lambda)\\
    \alpha \otimes \beta \ar@{|->}[rr] & &
        \alpha\text{"cup"}\beta}.$$ 
