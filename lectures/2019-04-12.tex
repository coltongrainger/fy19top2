\begin{quote}
\textit{%
    ``We'll have a reading course in the Fall to introduce stable/chromatic homotopy theory.''
}
\end{quote}

Today, $\Lambda$ is a commutative unital ring. We'll aim to define the cross product and cup product on cohomology.
% this particular day is fraught with BAD typos 
% especially given the sign conventions
% https://mathoverflow.net/questions/53315/references-for-sign-conventions-in-homological-algebra

\begin{defn}[Internal hom of chain complexes, Schreiber]
    \label{defn:internal_hom}
    Let $A$ and $B$ be chain complexes.
    Define a chain complex $\bkt{A,B}$ to have components
    \begin{equation*}
        [A,B]_n := \prod_{i \in \Z} \Hom_{\Mod}(A_i, B_{i+n})
    \end{equation*}
    (the collection of \term{homological degree}-$n$ maps between the underlying graded modules) 
    and whose differential is defined on homogeneously graded elements 
    $f \in [A,B]_n$ by
    \[\delta f = \partial_B f - (-1)^{n} f \partial_A.\]
        % This is Urs Schreiber's definition 2.63 at \url{https://ncatlab.org/schreiber/show/Introduction+to+Homological+Algebra#ChapterOnChainComplexes}.
    This complex is variously denoted 
    \[[A,B]_n = \Hom(A_\bullet, B_\bullet)_n  = \mathrm{hom}(A, B) = \{f_i \colon A_i \to B_{i+n}\}_{i \in \Z}.\]
    (Dually, maps of \term{cohomological degree}-$n$ belong to the chain complex $[A, B]_{-n} = [A, B]^{n}$.
    Note the differential $\delta f$ is well defined regardless of whether we're working with homological or cohomological degree.)
\end{defn}

\begin{quote}
\textit{%
    ``One algebraic way to motivate this is to observe that the signs in the differential for the Hom are precisely what is needed for 0-cycles in the $\hom(A,B)$ complex to be the set of morphisms of complexes $A\to B$ (and also, that the 0th homology group $H_0(\hom(A,B))$ is the set of homotopy classes of morphisms $A\to B$). This is quite great. Once you decide you want this, all the other signs you mention follow because you need various things to hold. For example, you want the adjunction between $\hom$ and $\otimes$ to hold for the internal versions, so this forces you to add signs to the $\otimes$, and so on.'' --Mariano Suárez-Álvarez, \url{https://math.stackexchange.com/questions/40468/}
}
\end{quote}

\begin{todo}[Internal hom is a functor, Schreiber]
    \label{todo:internal_hom}
    \hfill
    \begin{enumerate}
        \item Verify $[\bl,\bl] \colon \Ch \times \Ch \to \Ch$ is the internal hom functor on $\Ch$.
        \item Verify a map $f \in [A,B]$ of homological degree $0$ is a cycle if and only if $f$ is a chain map.
        \item Verify a map $h \in [A,B]$ of homological degree $1$ such that $\delta h = f - g$ is a chain homotopy from $g$ to $f$.
        \item Verify ${H}_{0}\paren{[A,B]}$ is the group of homotopy-equivalence classes of chain maps in $[A,B]_0$.
    \end{enumerate}
\end{todo}

\begin{quote}
\textit{%
    ``If one thinks of chain complexes as algebraic analogues of topological spaces and internal hom as an algebraic analogue of the internal hom in a nice category of topological spaces, then 0-cycles are analogues of points and 1-cycles are analogues of homotopies between points, so a 0-cycle in the internal hom is a continuous map and a 0-cycle up to boundaries is a homotopy class of continuous maps.'' --Qiaochu Yuan, \url{math.stackexchange.com/questions/40468/}
}
\end{quote}

\begin{coro}
    Say $G$ is a chain complex concentrated at degree $0$. Then the cohomological degree of \[f \colon A_p \to G_0\] is $p$.
    Since $\partial_G = 0$ (because it maps out of the concentration in degree $0$), we see that the chain complex $[A, G]$  has differential $\delta f = (-1)^{\deg f + 1} f \partial_A$.
\end{coro}

\begin{note}[Concentrated complexes and flat resolutions]
    For any PID $R$ and finitely generated $M \in \cat{RMod}$, the finite flat resolution $F_\bullet(M)$ of $M$ is quasi isomorphic to the complex $M$ concentrated in degree $0$:
    \begin{equation*}
        \begin{gathered}
            \xymatrix@C=1em{
                \cdots \ar[r] & 0 \ar[r] \ar[d] & F_1 \ar[r] \ar[d] & F_0 \ar[d] \\
                \cdots \ar[r] & 0 \ar[r]        & 0 \ar[r]          & M
            }
        \end{gathered}
        \qq{is quasi-isomorphic if and only if}
        \begin{gathered}
            \xymatrix@C=1em{
                0\ar[r] & F_0 \ar[r] & F_1 \ar[r] & M \ar[r] & 0
            }
        \end{gathered}
        \qq{is exact.}
    \end{equation*}
\end{note}

To handle products (especially products of CW-complexes) in singular homology,
we proved there exists a natural bilinear map $\times$ (the cross product)
from the product of two singular chain complexes to the chain complex of the
product of two spaces. With $\times$, we computed the boundary map of the chain
complex ${\Delta}_{*}\paren{X \times Y}$ using ${\Delta}_{*}\paren{X} \times {\Delta}_{*}\paren{Y}$ 
and the \term{degree of incidence} between $n$ and $n-1$ cells. 
Then, taking the interval $I = Y$ as the second space gave proof that (CW homology, and thus) singular homology satisfied the homotopy axiom.

Now we seek to define the cross product $\times$ on cochain complexes. (\TODO\ Motivate.) Consider the cochains $f\colon \Delta_p (X) \to \Lambda$ and $g \colon {\Delta}_{q}\paren{X} \to \Lambda$. There's a map induced by the ring structure
\[f \otimes g \colon \Delta_p(X) \otimes \Delta_q(Y) \to \Lambda \otimes \Lambda.\]
The tensor product is over $\Lambda$, so multiplication $\Lambda \otimes \Lambda \xrightarrow{m} \Lambda$ gives an isomorphism. Composing the previous two maps, for any $p$- and $q$-simplices $\sigma$ and $\tau$,
\[\sigma \otimes \tau \mapsto f(\sigma) \otimes g(\tau) \mapsto f(\sigma) g(\tau).\] 

Recall also, from the Eilenberg-Zilber theorem, there's a natural chain equivalence $\theta \colon \Delta_*(X \times Y) \to \Delta_*(X) \otimes \Delta_*(Y)$. (The content of this theorem was in constructing a chain homotopies $\theta \circ \times \simeq \id$).

\begin{defn}[Cross product on cochain complexes]
Define the cross product
\[\times \colon \Delta^p (X; \Lambda) \otimes \Delta^q (Y; \Lambda) \to \Delta^{p+q}(X\times Y; \Lambda),\]
by the rule
\[f \times g = f\otimes g \circ \theta.\]
\end{defn}

\begin{rem}[]
    Yu asked how 
    \[\theta \colon \Delta_p(X\times Y; \Lambda) \to (\Delta_*(X) \otimes \Delta_*(Y))_p\] 
    could be well defined. 
    \TODO\ Write $f \colon \Delta_p(X) \to \Lambda$, then extend to all of $\Delta_*(X)$ (by varying $p$?).
\end{rem}

\begin{lem}[]
$\delta (f \times g)  = \delta f \times g + (-1)^{\deg f} f \times \delta g$.
\end{lem}

\begin{proof}
Acyclic models.
\end{proof}

\begin{prop}
    \label{crossproductoncohomology}
    There's a natural (linear map out of the) product of homology groups
        \[\times \colon H^p(X; \Lambda) \otimes H^q(Y; \Lambda) 
            \to H^{p+q}(X \times Y; \Lambda)\]
    such that 
        \[[f] \otimes [g] \mapsto [f] \times [g],\]
    which is induced by
        \[\xymatrix{\Delta_*(X \times Y) \ar[r]^-\theta &
            \Delta_*(X) \otimes \Delta_*(Y) \ar[r]^-{f\otimes g} &
            \Lambda
        }.\]
\end{prop}

\begin{proof}
Let the unit $1 \in H^0(X; \Z)$ be the class of the augmentation $\epsilon \colon \Delta_0(X) \to\Z$. We appeal to the universal property of the product space $X \times Y$. The component projections induce chain maps on the cochain complexes.
\begin{equation*}
    \begin{gathered}
        \xymatrix{
        X\times Y \ar[dr]_{\pi_Y} \ar[r]^{\pi_X} & X \\
        & Y 
        }
    \end{gathered}
    \quad
    \overset{\Delta^*(\bl)}{\rightsquigarrow}
    \begin{gathered}
        \xymatrix{
        {\Delta}^{*}\paren{X \times Y} & {\Delta}^{*}\paren{X}  \ar[l]_{\pi^*_X} \\
        & {\Delta}^{*}\paren{Y}  \ar[ul]^{\pi^*_Y}
        }
    \end{gathered}
\end{equation*}
    I claim that the maps $\pi^*_X$ and $\pi^*_Y$ descend to cohomology, i.e., that there're well-defined homomorphisms $H^*(X) \to H^*(X \times Y)$ sending $\alpha$ under $\pi^*_X$ to $\alpha \times 1$ and symmetrically $H^*(Y) \to {H}_{*}\paren{X\times Y}$ sending $\beta \mapsto 1 \times \beta$.

To see this, consider the ``slice map" %(\TODO\ are these over categories?)
\begin{equation*}
    \xymatrix@=0.7em{
        \Delta_p (X) \otimes \Delta_0(Y) \ar[r]^{\times} &
        \Delta_p(X \times Y) \ar[r]^-{{\pi_X}_\Delta} &
        \Delta_p(X) \\
        \tau \otimes y \ar@{|->}[r]                      &
        (\tau \times y)(x) = (\tau(z), y) \ar@{|->} [r]  &
        \tau
    }
\end{equation*}
Extend the slice map linearly to a map on chains, then obtain an induced map on cochains.
\end{proof}

\begin{ex}[The twist map]
\label{twist}
How can we compare $\alpha \times \beta \in H^*(X\times Y; \Lambda)$ to $\beta \times \alpha \in H^*(Y\times X; \Lambda)$? 

Consider the twist map $T \colon X \times Y \to Y \times X$ taking $(\alpha, \beta) \mapsto (\beta, \alpha)$, which induces a map on cohomology.
\begin{equation*}
\xymatrix{H^*(X\times Y; \Lambda) & \ar[l]^{T^*} H^*(Y\times X; \Lambda)}.
\end{equation*}
\end{ex}

If $X = Y$, we apply proposition \ref{crossproductoncohomology} and pull back the diagonal map $d \colon X \to X \times X$ (where $x \mapsto (x,x)$) to obtain the cup product. 

\begin{defn}[Cup product]
    \label{defn:cup_product}
    The \term{cup product} is the bilinear map
    \begin{equation*}
        \xymatrix@R=0.7em{
            H^*(X; \Lambda) \otimes H^*(X; \Lambda)  \ar[r]^-{\times} & 
            H^*(X\times X; \Lambda) \ar[r]^-{d^*} & 
            H^*(X; \Lambda)\\
            \alpha \otimes \beta \ar@{|->}[rr] & &
            \alpha\smile\beta},
    \end{equation*} 
    given by the pullback of the diagonal map
    \begin{equation*}
        d^* \colon {H}^{*}\paren{X \times X}\to {H}^{*}\paren{X}
    \end{equation*}
    precomposed with the cross product $\times$.
    (To compute the cup product on cochains $f^p$ and $g^q$ in ${\Delta}^{p}\paren{X}$ and ${\Delta}^{q}\paren{X}$, see definition \ref{cupproductoncochains}.)
\end{defn}

\begin{todo}[]
From example \ref{twist}, prove that the cup product is a graded-commutative operation.
\end{todo}
