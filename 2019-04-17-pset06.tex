\documentclass[onesided]{ccg-pset}

\course{Math 6220}
\psnum{6}
\author{Colton Grainger}
\date{\today}

\begin{document}


\maketitle

\begin{enumerate}
\item 
    \begin{enumerate}
        \item Chapter V, Section 7, 284: 5. \textit{If $H_*(X)$ is finitely generated, then 
            \[
                \chi(X) = \sum (-1)^i \dim H_i(X; \Lambda) \qq{for any field $\Lambda$.}
           \]}
       \item Chapter VI, Section 1, 321: 3. \textit{For spaces $X$, $Y$ of bounded finite type, 
           \[
               \chi(X \times Y) = \chi(X)\chi(Y).
           \]}
    \end{enumerate}

\newpage

 \begin{note}[See Chapter VI, Section 4 and Example 4.12.]
    By a \emph{graded commutative ring}, we will mean a graded abelian group $R^*$ together with a homomorphism of graded abelian groups $\mu \colon R^* \otimes R^* \to R^*$ such that,
    \begin{itemize}
       \item There exists $1 \in R^0$ which is a two sided unit for $\mu$, and
       \item $\mu (a\otimes b) = (-1)^{\deg(a)\deg(b)} \mu(b\otimes a)$.
    \end{itemize}
    The cup product gives a graded commutative ring structure on the cohomology of a space. 
\end{note}

\item Chapter VI, Section 1, 321: 2. 
    \textit{Let $X_p$ be the space resulting from attaching an $n$-cell to $S^{n-1}$ by a map of degree $p$. Use the Künneth Theorem to compute the homology of $X_p \times X_q$ for any $p, q$.}


\newpage

\item 
    \begin{enumerate}
        \item \textit{Write down the ring structure of $H^*(S^n)$ and of $H^*(S^n \times S^m)$.}        
        \item \textit{We will see later that $H^*(\R P^n ; \Z/2) \cong (\Z/2)[x]/\ang{ x^{n+1} }$ for $x \in H^1(\R P^n ; \Z/2)$. Use this to prove that $\R P^3$ is not homotopy equivalent to $\R P^2 \vee S^3$.}
        \item Chapter VI, 334: 3. \textit{Show that any map $S^4 \to S^2 \times S^2$ must induce the zero homomorphism on $H_4(\bl)$.}
\end{enumerate}

\newpage

% \begin{defn*}[Alexander--Whitney diagonal approximation]
%    Let $\sigma \colon \Delta_n \to X$ be a singular $n$-simplex in $X$. The \term{Alexander--Whitney diagonal approximation} explicitly computes the image of $\sigma$ under the chain map $\Delta \colon \Delta_*(X) \to \Delta_*(X) \otimes \Delta_*(X)$ from the \term{front and back faces} of $\sigma$.
%    \[
%        \Delta \sigma  = \sum_{p + q = n} \norm{ \sigma }_\text{front}^p \otimes \norm{ \sigma }_\text{back}^q
%    .\]
% \end{defn*}

% \begin{prop*}[Computing the cup product]
%     Say $f$ and $g$ are in the cochains with degrees $p$ and $q$ respectively, such that $p+q = n$. Then
%     \begin{align*}
%         (f \smile g)(\sigma) &= (f \otimes g)(\Delta\sigma)\\
%                            &= (f \otimes g) \paren{\sum_{i+j = n} \norm{ \sigma }_\text{front}^i \otimes \norm{ \sigma }_\text{back}^j} \\
%                            &= (f \otimes g) \paren{ \norm{ \sigma }_\text{front}^p \otimes \norm{ \sigma }_\text{back}^q}  \\
%                            & = (-1)^{\deg g \deg f} f(  \norm{ \sigma }_\text{front}^p ) g(\norm{ \sigma }_\text{back}^q) \quad (\text{an element of } \Lambda).
%     \end{align*}
% \end{prop*}

\item Chapter VI, 334: 5.
      \textit{Any two chain maps $\alpha, \beta \colon \Delta_*(X) \to \Delta_*(X\otimes X)$ that agree with the diagonal approximation
      \[
          \Delta(x) = x \otimes x \qq{in the $0$th degree}
      \]
      are chain homotopic: $\alpha \simeq \beta$.}

\begin{proof}

   Let $X$ be a topological space. We will manipulate the functor:
   \begin{equation}
      \xymatrix{
         \Top \ar[rr]^{\mathrm{Maps}(\Delta^n_\Top, \bl)} \ar@/_2pc/[rrrr]^{\Delta_*(\bl)}  
         & & \cat{sSet} \ar[r]^{\Z\ang{\bl}}
         & \cat{sAb} \ar[r]^{C(\bl)} 
         & \cat{Ch}_*^+}
   \end{equation} 

   that eventually\footnote{Here $\mathrm{Maps}(\Delta^n_\Top, \bl)$ takes the space $X$ to the simplicial set $\mathrm{Sing}X$, which $\Z\ang{\bl}$ takes to the simplicial (free) abelian group $\Z \ang{\mathrm{Sing}X}$, which then $C(\bl)$ takes to the alternating face map complex $\Delta_*(X)$.} sends $X$ to its singular chain complex $\Delta_*(X)$.

   So, let $\alpha, \beta \colon \Delta_*(X) \to \Delta_*(X) \otimes \Delta_*(X)$ be chain homomorphisms such that $\alpha = \beta$ on $\Delta_0(X)$:
   \[
      \alpha(x) = \beta(x) =  x \otimes x \qq{for $x \colon \Delta_{0} \to X$.}
   \]
   To see that $\alpha \simeq \beta$ are chain homotopic, we need to construct a sequence of homomorphisms
   \[
      \set{h_n \colon \Delta_n(X) \to \paren{\Delta_*(X) \otimes \Delta_*(X)}_{n+1}}_{n \in \Z_{\ge 0}}
   \]
   \newcommand{\sing}[1]{\mathrm{Sing}}
   \newcommand{\sich}[2]{\Delta_{#1}(#2)}
   such that, for each $n \in \Z_{\ge 0}$ and each singular simplex $\sigma \in \sich n X$, the homomorphisms $h_{n}$ and $h_{n-1}$ satisfy the homotopy condition
   \begin{equation}
      \label{hocondition}
      \paren{\alpha_n - \beta_n}(\sigma) = \paren{\delta_{n+1} h_n - h_{n-1} \partial_n}(\sigma),
   \end{equation}
   which is an equality of chains in $\paren{\sich * X \otimes \sich * X}_n$.

   For the \textsc{base case}, let $X$ be \emph{any topological space} and $\alpha$, $\beta$ \emph{any diagonal approximations}. Define \[h_0 \colon \sich 0 X \to \sich 0 X \otimes \sich 0 X\] by $h_0 (x) = 0$. Then (trivially) \[0 = \alpha_0 - \beta_0 = \delta_1 h_0 - h_{-1} \partial_0\] as $\alpha$ and $\beta$ are diagonal approximations that agree on $0$-simplices.

   For the \textsc{inductive step}, let $X$, $\alpha$, and $\beta$ be as above, and say the first $p<n$ homomorphisms
   \[
      \set{h_p \colon \Delta_p(X) \to \paren{\Delta_*(X) \otimes \Delta_*(X)}_{p+1}}_{p < n}
   \]
   and satisfy the homotopy condition \eqref{hocondition}. 

   We will complete the inductive step finding a chain homotopy $\eta$ on an {acyclic model}, $\Delta_n$. Consider the identity map $\iota_n \colon \Delta_n \to \Delta_n$ from the topological $n$-simplex $\Delta_n$ to itself. Note $\iota_n$ is both a singular $n$-simplex 
   \[\iota_n \in \sing \Delta_n := \mathrm{Maps}(\Delta^n_\Top, \Delta_n),\] and a continuous map between topological spaces 
   \[
      \Delta_n \xrightarrow{\iota_n} \Delta_n.
   \]
   In particular, \[\delta \iota_n \in \Delta_{n-1}(\Delta_n)\] is a $n-1$-chain. 
   So say that $A$ and $B$ are diagonal approximations for the singular chain complex $\sich * \Delta_n$.
   By the inductive hypothesis, the homomorphisms $\eta_p$ for $p < n$ are defined. Therefore
   \begin{align*}
      \label{indhypo}
      & \paren{A_{n-1} - B_{n-1}}(\partial\iota_n) = \paren{\delta_{n} \eta_{n-1} - \eta_{n-2} \partial_{n-1}}(\partial \iota_r) \\
      & \iff \paren{A_{n-1} - B_{n-1} - \eta_{n-2} \partial_{n-1}}(\partial\iota_n) =  \paren{\delta_{n} \eta_{n-1}}(\partial \iota_r) \\
      & \iff \paren{A_{n-1} - B_{n-1}}(\partial\iota_n) =  \paren{\delta_{n} \eta_{n-1}}(\partial \iota_r) 
   \end{align*}
   is an equality of chains in $\paren{\sich * \Delta_r \otimes \sich * \Delta_n}_{n-1}$. 

   Computing the boundary of $\ldots$ (we finished this proof in lecture.)
\end{proof}

\end{enumerate}

\end{document}
