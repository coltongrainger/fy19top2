\documentclass[10pt]{amsart}
\usepackage{fullpage}
% https://tex.stackexchange.com/questions/174073
\usepackage{parskip}

% macros
\usepackage{ccg-macros}
\usepackage{amscd, amssymb}
\usepackage{physics}

% pandoc is utf8 native
\usepackage[utf8]{inputenc}
\usepackage[T1]{fontenc}

% extra macros

% pandoc transfers h2 to subsections, I prefer h2 for sections
\let\subsubsection\subsection
\let\subsection\section
\let\section\chapter
\let\chapter\part

% pandoc section numbering
\setcounter{secnumdepth}{0}

% pandoc urls as footnotes
\usepackage[unicode=true]{hyperref}
\renewcommand{\href}[2]{#2\footnote{\url{#1}}}


% pandoc for upquote if available
\IfFileExists{upquote.sty}{\usepackage{upquote}}{}

% tables, pandoc manges export





\usepackage[all]{xy}

% pandoc prevent overfull lines
\setlength{\emergencystretch}{3em}  
\providecommand{\tightlist}{%
  \setlength{\itemsep}{0pt}\setlength{\parskip}{0pt}}

% pandoc default figure placement to htbp
\makeatletter
\def\fps@figure{htbp}
\makeatother

\title{Incidence numbers in Cellular Homology}
\author{Colton Grainger (scribe) \and Shen Lu (presenter)}
\date{2019-03-01}

\begin{document}

\maketitle



\providecommand{\ccw}{C^{\text{CW}}}
\providecommand{\k}[1]{K^{( #1 )}}
\renewcommand{\deg}[1]{\operatorname{deg}\left( #1 \right)}
\newcommand{\defeq}{\stackrel{\text{def}}{=}}

This problem is set from Bredon {[}1, No. IV.11.12{]}. It demonstrates

\begin{itemize}
\tightlist
\item
  for a CW-complex \(K\), the differential \(\beta\) of the chain
  complex \(C^{\text{CW}}_*(K)\) satisfies \(\beta^2 =0\), and thus
\item
  for an \(n+1\) cell \(\sigma\) and an \(n-1\) cell \(\omega\), we've
  \(\sum_\tau [\omega : \tau][\tau : \sigma] = 0\) with \(\tau\) ranging
  over all \(n\)-cells.
\end{itemize}

\subsubsection{Given}

Let \(K\) be a CW-complex, with \(n\)-skeleton \(K^{( n )}\) for
\(n \ge 0\). Because \(K^{( n )}\) contains an open neighborhood around
the closed subset \(K^{( n-1 )}\) that deform retracts onto
\(K^{( n-1 )}\), we know:

\begin{itemize}
\tightlist
\item
  The relative homology \(H_*(K^{( n )}, K^{( n-1 )})\) is isomorphic to
  the reduced homology \(\tilde{H}_*(K^{( n )} / K^{( n-1 )})\).
\item
  The quotient space \(K^{( n )} / K^{( n-1 )}\) is homeomorphic to the
  wedge \(\vee ( I^n/\partial I^{n-1} ) \approx \vee S^n\), and thus
\end{itemize}

\begin{equation}
\xymatrix{
H_*(K^{( n )} , K^{( n-1 )}) \ar[r]^-{\cong}_-{e_*}
    & \tilde{H}_*(K^{( n )} / K^{( n-1 )}) \ar[r]^-{\cong}_-{\text{h.a.}}
    & \tilde{H}_*(\vee ( I^n/\partial I^{n-1} )) \ar[r]^-\cong
    & \displaystyle{ \bigoplus_\text{$n$-cells of $K^{( n )}$} } \tilde H_*(S^n).
}
\label{eq:ncells}
\end{equation}

We may define a chain complex \(C^{\text{CW}}_*(K)\) associated to \(K\)
as follows:

\begin{itemize}
\item
  Let the chain group \(C^{\text{CW}}_n(K)\) be
  \(H_n(K^{( n )}, K^{( n-1 )})\). This is the free abelian group (in
  the \(n\)th degree of the graded group) at the end of
  \eqref{eq:ncells} whose basis is the set of \(n\)-cells attached to
  \(K^{( n-1 )}\).
\item
  Let the differential
  \(\beta_n \colon C^{\text{CW}}_n(K) \to C^{\text{CW}}_{n-1}(K)\) be
  the composite

  \begin{equation}
    \xymatrix{
    C^{\text{CW}}_n(K) = 
    H_n(K^{( n )}, K^{( n-1 )}) \ar[r]^-{\delta_n} \ar@/_2.0pc/^-{\beta_n}[rr] 
        & H_{n-1}( K^{( n-1 )} ) \ar[r]^-{j_{n-1}} 
        & H_{n-1}( K^{( n-1 )},K^{( n-2 )}) = C^{\text{CW}}_{n-1}(K).
    }
    \label{eq:beta}
    \end{equation}
\end{itemize}

In \eqref{eq:beta}, the boundary map \(\delta_n\) arises from the long
exact sequence for the pair \((K^{( n )}, K^{( n-1 )})\), and the map of
relative homology groups
\(j_{n-1} \colon H_{n-1} (K^{( n-1 )}) \to H_{n-1} (K^{( n-1 )}, K^{( n-2 )})\)
is induced by the inclusion of skeleta
\(j \colon (K^{( n-1 )}, \emptyset) \hookrightarrow (K^{( n-1 )}, K^{( n-2 )}).\)

From lecture {[}2, No. 1.11.3{]}, we know \(\delta_n\) respects the
attaching maps; for an \(n\)-cell \(\sigma\) with attaching map
\(f_{\partial\sigma}\),
\[\delta_n [I^n_\sigma] = [f_{\partial \sigma} ( \partial I^n_\sigma )].\]

And so, the differential \(\beta_n\) can be described with ``incidence
numbers'' {[}3, No. 8.5{]}. For an \(n\)-cell \(\sigma\) and an \(n-1\)
cell \(\tau\), define

\begin{equation}
[\tau : \sigma] := \text{deg}\bigg( \xymatrix @R-1.8pc{
    S^{n-1} \ar[r]^{f_{\partial\sigma}} \ar@2{~}[d]
    & K^{( n-1 )} \ar[r] \ar@/_1.5pc/^-{p_\tau}[rrr]
    & K^{( n-1 )} / K^{( n-2 )} \ar[r]^-{\approx}
    & \vee S^{n-1} \ar[r]^{\text{find $\tau$}}
    & S^{n-1} \ar@2{~}[d]\\
    \partial I^n_\sigma & & & & I^{n-1}_\tau / \partial I^{n-1}_\tau
}\bigg).
\label{eq:incidencedef}
\end{equation}

To make three comments. First, we take for granted the rule
\(\sigma \mapsto \sum_\tau [\tau : \sigma] \tau\) on generators
\(\sigma\) in \(C^{\text{CW}}_n(K)\) extends linearly and \emph{is} the
differential \(\beta_n\) in \eqref{eq:beta}. See {[}1, p. 203{]}. So
write \(\beta_n(\sigma) := \sum_\tau [\tau : \sigma] \tau\). Second, all
but finitely many terms in the sum \(\sum_\tau [\tau : \sigma]\tau\)
must be zero. This is because the compact set \(\partial I^n_\sigma\) is
attached by \(f_{\partial \sigma}\) to a \emph{compact} subset of
\(K^{( n-1 )}\). Third, the projection \(p_\tau\) that ``finds''
\(\tau\) in \eqref{eq:incidencedef} is the unique map
\(p_\tau \colon K^{( n-1 )} \to S^{n-1}\) satisfying:

\begin{enumerate}
\def\labelenumi{\roman{enumi}.}
\tightlist
\item
  \(p_\tau \circ f_\tau = \gamma_{n-1} = \text{smash product $\gamma \wedge \cdots \wedge \gamma$ of $n-1$ copies of the quotient map $\gamma\colon I^1 \to S^1$}\)
\item
  \(p_\tau \circ f_{\tau'} = \text{constant map to base point, for $\tau' \neq \tau$}\).
\end{enumerate}

Now, we almost done setting up results and rehashing definitions needed
to make \(C^{\text{CW}}_*(K)\) a chain complex. It remains to argue that
the differential \(\beta\) is of order \(2\), i.e., that
\(\beta^2 = 0\). So consider the following three long exact sequences in
relative homology. (This is Ulrike Tillmann's argument {[}3, No.
8.6{]}.)

\begin{equation}
\xymatrix{
\cdots \ar[r]
    & H_{n+1}(K^{( n+1 )}, K^{( n )}) \ar[r]^-{\delta_{n+1}}
    & H_n( K^{( n )} ) \ar[r] \ar@{=}[dl]
    & H_n( K^{( n+1 )}) \ar[r]
    & \cdots\\
\cdots \ar[r] 
    & H_n(K^{( n )}) \ar[r]_-{j_n}
    & H_n(K^{( n )}, K^{( n-1 )}) \ar[r]^-{\delta_n}
    & H_{n-1} (K^{( n-1 )}) \ar[r] \ar@{=}[dl]
    & \cdots\\
& \cdots \ar[r]
    & H_{n-1}(K^{( n -1 )}) \ar[r]_-{j_{n-1}}
    & H_{n-1} (K^{( n-1 )}, K^{( n-2 )}) \ar[r]
    & \cdots
}
\label{eq:order2}
\end{equation}

Notice
\(\beta_{n} \beta_{n+1} = (j_{n-1}\delta_n) (j_n \delta_{n+1}) = j_{n-1} (\delta_n j_n) \delta_{n+1} = 0,\)
as \(\delta_n j_n = 0\) by exactness of the middle row.

\subsubsection{To prove}

Let \(K\) be a CW-complex. For all \(n+1\) cells \(\sigma\) and \(n-1\)
cells \(\omega\),
\begin{equation}\label{eq:boundaries}\sum_\tau [\omega : \tau][\tau : \sigma] = 0,\end{equation}
where \(\tau\) ranges over all \(n\)-cells.

\subsubsection{Proof}

We require \(\beta^2 = 0\), as in \eqref{eq:order2}. We also require
\(\beta(\sigma) = \sum_\tau [\tau: \sigma]\tau\), as discussed after
\eqref{eq:incidencedef}. Whence \begin{align*}
\beta^2(\sigma) & = \sum_\tau [\tau : \sigma] \beta(\tau) & \text{($\beta$ is linear)}\\
    & = \sum_\tau [\tau : \sigma] \sum_\omega [\omega: \tau]\omega & \text{(evaluate)}\\
    & = \displaystyle{\sum_{\forall \tau, \omega}[\omega : \tau][\tau : \sigma] \omega}. & \text{($\ZZ$ is a commutative ring)}
\end{align*} \(C^{\text{CW}}_{n-1}(K)\) is the free abelian group whose
basis is the set of \(n-1\) cells in \(K\). So if
\(\beta^2(\sigma) = 0\), then the coefficient
\([\omega:\tau][\tau:\sigma]\) of each \(n-1\) cell \(\omega\) had
better be zero. Thus, fixing \(\omega\), we conclude
\(\sum_{\tau}[\omega : \tau][\tau : \sigma] \omega = 0\). \qedsymbol

\subsubsection{Remarks}

\begin{enumerate}
\def\labelenumi{\roman{enumi}.}
\item
  Here's another way to remember \eqref{eq:boundaries} in the case that
  \(K\) is finite. Take the matrices \([\beta_{n+1}]\) and \([\beta_n]\)
  representing \(\beta_{n+1}\) and \(\beta_n\) with respect to the
  \emph{finite} bases for \(C^{\text{CW}}_{n+1}(K)\) and
  \(C^{\text{CW}}_n(K)\). Because \(\beta\) is order two,
  \([\beta_{n}][\beta_{n+1}] = 0\) and \eqref{eq:boundaries} follows
  from matrix multiplication.
\item
  If the coefficient \([\tau : \sigma]\) is the ``incidence'' of
  \(\sigma\) to \(\tau\), then the matrix \([\beta_n]\) suggests itself
  as the ``incidence matrix'' of the differential \(\beta_n\).
\item
  But the term ``incidence matrix'' is typically reserved for the
  following situation: Take the differential
  \(\beta_1 \colon C^{\text{CW}}_1(K) \to C^{\text{CW}}_0(K)\). How does
  the matrix \([\beta_1]\) describe the \emph{directed graph} whose
  vertices are \(0\)-cells in \(K^{( 0 )}\) and whose directed edges are oriented
  \(1\)-cells in \(K^{( 1 )}\)?
\end{enumerate}

\subsection*{References}
\addcontentsline{toc}{subsection}{References}

\hypertarget{refs}{}
\leavevmode\hypertarget{ref-Bre93}{}%
{[}1{]} G. E. Bredon, \emph{Topology and Geometry}. New York:
Springer-Verlag, 1993.

\leavevmode\hypertarget{ref-Bea19}{}%
{[}2{]} A. Beaudry, ``Math 6220 Class Notes.'' 2019.

\leavevmode\hypertarget{ref-Til13}{}%
{[}3{]} U. Tillmann, ``Algebraic Topology Lecture Notes.'' 2013.


\end{document}
