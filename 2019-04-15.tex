\begin{quote}
   \textit{``Does the torus $S^1 \times S^1$ have a cohomology ring\footnote{\textsf{Hint.} The cohomology ring for $S^1$ is to $\RP^2$ as $\bl$ is to $\bl$ (?) with spheres adjoined (!?). \textsf{Moral.} Consider the homogeneous elements of the ring. Just describe the ring. For example, when algebraic topologists write $\Z[x]/\ang{x^4}$, it is understood to be \emph{a homogeneous ring}, with addition \emph{strictly levelwise} but with products (via the cup product) allowed between non-homogeneous degrees.} 
   that's not a polynomial ring?''}
\end{quote}


\newcommand{\coho}[2]{H^*\paren{ #1; #2 }}

\textsc{Last time}, we defined (extended actually) the cross product
\[
    \times \colon H^*\paren{ X; \Lambda } \otimes \coho{Y}{\Lambda} \to \coho{X\times Y}{\Lambda}
,\]

which arises from the map on cochains (say $f\colon A_* \to \Lambda$ and $g\colon B_* \to \Lambda$) given by
\begin{align*}
    f \otimes g \colon  A_* \otimes B_* & \to \Lambda\\
    (f \otimes g)(a \otimes b) &= (-1)^{\deg a \deg g} f(a) g(b).
\end{align*}

\renewcommand{\ev}{\mathrm{ev}}

\begin{defn}
   Let $X \in \Top$ and $G \in \Ab$. Shouldn't there be an \term{evaluation} $\ev$ from chain complex of $p$-cochains $f \colon \Delta_p(X) \to G$ tensored with the chain complex $\Delta_p(X)$ of $p$-chains? 
    \[
        \Delta^p(X; G) \otimes \Delta_p(X) \xrightarrow{\ev} G
    \]
    In fact, there is such an evaluation:
    \[
     \ev\colon f \otimes c \mapsto f(c).
    \] 
\end{defn}

We'll see that $\ev$ induces a map on cohomology, which is denoted by the brackets (in Halmos style) 
\[
   \ang{\abs{f}, \abs{ c }} \in G.
\]

\begin{lem}[Kronecker pairing]
   The evaluation $\Delta^p(X; G) \otimes \Delta_p(X) \xrightarrow{\ev} G$ induces a homomorphism out of the graded ring
   \begin{equation}
       \label{kpairing}
       H^p(X;G) \otimes H_p(X) \to G
   \end{equation}
   such that $\ang{\abs{f}, \abs{ c }}\mapsto f(c)$.
\end{lem}

\begin{proof}
The \term{Kronecker pairing} is the argument that $f(c)$ does not depend on representatives $f$ or $c$ (from the cochain, resp, chain complexes). Consider that in the proof of the universal coefficient theorem, we found a map $\beta$ from $\coho{X}{G}$ to $\Hom(H_p(X), G)$ such that $\abs{f} \mapsto \set{ \abs{c} \mapsto f(c) }$ gave a group homomorphism. Use this.
\end{proof}

\begin{defn}[Cap product]
   The \term{cap product} operation over a topological space $X$ is the above~\eqref{kpairing} pairing, which is ``given by combining the \term{Kronecker pairing} of the cohomology class with the image of the homology class under diagonal and using the Eilenberg-Zilber theorem.'' (See \url{https://ncatlab.org/nlab/show/cap+product}.)
\end{defn}

\begin{prop}[Computing with cochains]
   For the singular chain complexes over a topological space $X$, let $f, g$ be cochains in $\Delta^*(X)$ and $\alpha, \beta$ chains in $\Delta_*(X)$. Then explicitly, 
    $(f\otimes g)(\alpha \times \beta) = (-1)^{\deg \alpha \deg g} f(\alpha)g(\beta)$.
\end{prop}

\begin{proof}
\TODO. Verify that the equation
\[
    \delta(f \smile g) = \delta f \smile g + (-1)^{\deg}f \smile \delta g
\]
follows from the boundary formula for the cross product $\times$.
\end{proof}

\begin{defn}[Cup product]
   For $X \in \Top$, the diagonal map $d \colon X \to X \times X$ yields a chain map $d_\Delta \colon \Delta_*(X) \to \Delta_*(X)$ such that 
   \[
      d_\Delta \qq{precomposed with} \theta \qq{(from Eilenberg-Zilber) is the natural \term{diagonal approximation}} \Delta.
   .\]
   Schematically,
   \[
      \xymatrix{\Delta_*(X) \underset{\inj}{d_\Delta} \Delta_*(X \times X) \ar[r]^\theta 
                  & \Delta_*(X) \otimes \Delta_*(X)}
   .\]
   The \term{cup product} of two homogeneous cochains $f$ and $g$ is defined to be \[f \smile g = (f \otimes g) \theta d_\Delta.\]
\end{defn}

\begin{todo}[Chain maps on $\Delta_*(X) \to \Delta_*(X\otimes X)$]
    Any two chain maps $\Phi, \Psi \colon \Delta_*\to \Delta_*(X\otimes X)$ which agree with 
    \[
        \Delta(x) = x \otimes x \qq{in the $0$th degree}
    \]
    are chain homotopic.  
\end{todo}

\begin{defn}[Alexander Whitney Diagonal Approximation Map]
    Let $\sigma \colon \Delta_n \to X$ be a singular $n$-simplex in $X$. Then there's a map (from the front and back faces \TODO)
   \[
       \Delta \sigma  = \sum_{p + q = n} \norm{ \sigma }_\text{front}^p \otimes \norm{ \sigma }_\text{back}^q
   .\]
\end{defn}

Say $f$ and $g$ are in the cochains with degrees $p$ and $q$ respectively, such that $p+q = n$. Then
\begin{align*}
    (f \smile g)(\sigma) &= (f \otimes g)(\Delta\sigma)\\
                       &= (f \otimes g) \paren{\sum_{i+j = n} \norm{ \sigma }_\text{front}^i \otimes \norm{ \sigma }_\text{back}^j} \\
                       &= (f \otimes g) \paren{ \norm{ \sigma }_\text{front}^p \otimes \norm{ \sigma }_\text{back}^q}  \\
                       & = (-1)^{\deg g \deg f} f(  \norm{ \sigma }_\text{front}^p ) g(\norm{ \sigma }_\text{back}^q) \in \Lambda
\end{align*}

To laundry list facts about the cup product.
\begin{enumerate}
    \item The cup product is natural. If I have a map $\phi \colon X \to Y$ in $\Top$, then $\phi^*(\alpha \smile \beta) = \phi^*(\alpha) \smile \phi^*(\beta)$.
    \item $\alpha \smile 1 = \alpha = 1 \smile \alpha$, where $1$ is the class of the augmentation.
    \item $\smile$ is associative.
    \item $\alpha \smile \beta = (-1)^{\deg \alpha \deg \beta} \beta \smile \alpha$ is (anti)commutative.
\end{enumerate}
 
\begin{todo}[A derivation from the cup product]
Let $A, B  \subset X$ in $\Top$ such that $A, B$ are open in $X$ (this allows us to get $\Delta_*(A) + \Delta_*(B) \surj \Delta_*(A \smile B)$). Then
        \[
            \coho{X, A}{\Lambda} \otimes \coho{X, B}{\Lambda} \to \coho{X, A}{\Lambda}
        \]
        with (convince oneself! from the snake lemma for the exact sequence on cochains) a induced long exact sequence
        \[
            \xymatrix{\ldots & \ar[l] H^{*+1}(X, A; \Lambda) & \ar[l]^\delta H^*(A, \Lambda) & \ar[l]^{i^*} H^*(X) & \ar[l] \ldots}
        \]
        such that $\delta$ is a derivation (?) \TODO\$\delta^*(\alpha \smile i^*(\beta)) = \alpha \smile \delta^*(\beta)$.
% this is wrong:  <ccg, 2019-04-15> %
\end{todo}
