\documentclass[10pt]{amsart}
\usepackage{fullpage}
% https://tex.stackexchange.com/questions/174073
\usepackage{parskip}

% macros
\usepackage{ccg-macros}
\usepackage{amscd, amssymb}
\usepackage{physics}

% pandoc is utf8 native
\usepackage[utf8]{inputenc}

% pandoc transfers h2 to subsections, I prefer h2 for sections
\let\subsubsection\subsection
\let\subsection\section
\let\section\chapter
\let\chapter\part

% pandoc section numbering
\setcounter{secnumdepth}{5}

% pandoc urls as footnotes
\usepackage[unicode=true]{hyperref}
\renewcommand{\href}[2]{#2\footnote{\url{#1}}}


% pandoc for upquote if available
\IfFileExists{upquote.sty}{\usepackage{upquote}}{}

% tables, pandoc manges export





\usepackage[all]{xy}

% pandoc prevent overfull lines
\setlength{\emergencystretch}{3em}  
\providecommand{\tightlist}{%
  \setlength{\itemsep}{0pt}\setlength{\parskip}{0pt}}

% pandoc default figure placement to htbp
\makeatletter
\def\fps@figure{htbp}
\makeatother

\title{First look at the Universal Coefficient Theorem}
\author{Colton Grainger (scribe) \and Marvin Qi (presenter)}
\date{2019-02-13}

\begin{document}

\maketitle



This problem is set from Bredon {[}1, No. IV.5.1{]}; it works out
techniques that might later play into a proof of the universal
coefficient theorem.

\emph{Given}. Let \(p \colon \ZZ \to \ZZ\) be multiplication by the
prime \(p\), and consider the short exact sequence \begin{equation}
0 \to \ZZ \xrightarrow{p} \ZZ \xrightarrow{\pi} \ZZ_p \to 0.
\label{eq:p}
\end{equation}

Say \(X\) is a topological space. We assume
\(\Delta_*(-) \colon \mathsf{Top} \to \mathsf{Comp}\) is the singular
chain functor from topological spaces to chain complexes of abelian
groups, with \(H_*(-) \colon \mathsf{Comp} \to \mathsf{GradedAb}\) the
homology functor on chain complexes. For an abelian group \(G\), denote
by \(H_*(C_*; G)\) the \emph{homology of \(C_*\) with coefficients in
\(G\)}, which is the homology of the chain complex \(G \otimes C_*\)
with boundary operator \(\id_G \otimes \partial\).

\emph{To prove.} There is an exact sequence \begin{equation}
0 \to \ZZ_p \otimes H_n(X) \to H_n(X;\ZZ_p) \to \Tor(H_{n-1}(X), \ZZ_p) \to 0,
\label{eq:wts}
\end{equation} with a homomorphism
\(H_n(X; \ZZ_p) \to \ZZ_p \otimes H_n(X)\) such that \eqref{eq:wts}
splits.

\emph{Proof.} We'll construct the exact sequence \eqref{eq:wts}, then
the splitting homomorphism.

Each group in \(\Delta_*(X)\) is free abelian, and, in particular, flat
as a \(\ZZ\)-module. Thus tensoring the short exact sequence
\eqref{eq:p} on the right by the \(n\)th singular chain \(\Delta_n(X)\)
group preserves exactness, and produces: \begin{equation}
\xymatrix{
0 \ar[r] & \ZZ\otimes \Delta_n(X) \ar[r]^{p\otimes \id_\Delta} & \ZZ\otimes \Delta_n(X) \ar[r]^{\pi\otimes \id_\Delta} & \ZZ_p \otimes\Delta_n(X) \ar[r]&  0
}
\label{eq:level}
\end{equation}

Lining up copies of \eqref{eq:level} levelwise with the appropriate
differentials, we exhibit the short exact sequence of chain complexes:
\begin{equation}
\xymatrix{
& \vdots \ar[d] & \vdots \ar[d] & \vdots \ar[d] &\\
0 \ar[r] & \ZZ\otimes \Delta_n(X) \ar[d]^{\id_\ZZ \otimes \partial_n} \ar[r]^{p\otimes \id_\Delta} & \ZZ\otimes \Delta_n(X) \ar[d]^{\id_\ZZ \otimes \partial_n} \ar[r]^{\pi\otimes \id_\Delta} & \ZZ_p \otimes\Delta_n(X) \ar[d]^{\id_{\ZZ_p} \otimes \partial_n}\ar[r]&  0\\
0 \ar[r] & \ZZ\otimes \Delta_{n-1}(X) \ar[d] \ar[r]^{p\otimes \id_\Delta}  & \ZZ\otimes \Delta_{n-1}(X) \ar[d] \ar[r]^{\pi\otimes \id_\Delta} & \ZZ_p \otimes\Delta_{n-1}(X) \ar[d] \ar[r]&  0\\
& \vdots & \vdots & \vdots  &\\
}
\label{eq:ch1}
\end{equation}

As abelian groups, we'll identify
\(\ZZ \otimes \Delta_n(X) = \Delta_n(X)\) and
\(\ZZ_p \otimes \Delta_n(X) = \Delta_n(X)/p\) (the cokernel of the
\(p\)th multiple map). Accordingly, we'll write the homomorphisms
\(p \otimes \id_\Delta = p\), \(\pi \otimes \id_\Delta = \pi\), with all
the differentials \(\id \otimes \partial_n\) as \(\partial\). Given
these abbreviations, the short exact sequence \eqref{eq:ch1} becomes:
\begin{equation}
\xymatrix{
0 \ar[r] & \Delta_*(X) \ar[r]^{p} &\Delta_*(X) \ar[r]^{\pi} & \Delta_*(X)/p \ar[r]&  0
}
\label{eq:ses}
\end{equation}

By inspection of the definition of homology with coefficients along with
the diagram \eqref{eq:ch1}, the long exact sequence of homology groups
induced by \eqref{eq:ses} is: \begin{equation}
\xymatrix{
\cdots \ar[r]^{\delta_*} & H_n(X) \ar[r]^{ p_* } & H_n(X) \ar[r]^{ \pi_* } & H_n(X; \ZZ_p) \ar[r]^{\delta_*} & H_{n-1}(X) \ar[r]^{p_*} & \cdots
}
\label{eq:les}
\end{equation}

We now make an aside to argue the algebraic structure of a long exact
sequence is encoded in a series of intertwined short exact sequences
{[}2, No. 2.5{]}. Start by considering an exact sequence of abelian
groups
\[\cdots \to A_{n+2} \to A_{n+1} \to A_{n} \to A_{n-1} \to A_{n-2} \to \cdots.\]
By definition of exactness at \(A_n\),
\begin{equation}\label{eq:defexact}\im (A_{n+1} \to A_n) = \ker(A_n \to A_{n-1}).\end{equation}
Quotienting by the kernel of the map out of \(A_n\), we have an
isomorphism onto the image in \(A_{n-1}\) \begin{equation}
A_n/\ker (A_n \to A_{n-1}) \cong \im(A_n \to A_{n-1}) 
\label{eq:isomthm}
\end{equation} Now, the cokernel of an additive homomorphism is the
target group quotiented by the incoming image. Substituting
\eqref{eq:defexact} into \eqref{eq:isomthm} gives an example of such
\[\coker ( A_{n+1} \to A_n) := A_n/\im (A_{n+1} \to A_n) \cong \im(A_n \to A_{n-1}).\]
Putting it all together, there's a group \(C_n\) that is

\begin{itemize}
\tightlist
\item
  the kernel of the map \(A_n \to A_{n-1}\),
\item
  isomorphic to the image of the map \(A_{n+1} \to A_n\), and
\item
  isomorphic to the cokernel of the map \(A_{n+2} \to A_{n+2}\).
\end{itemize}

Moreover, there are groups \(C_{n+k}\) for \(k \in \ZZ\) satisfying
analogously conditions. We may now decorate our long exact sequence with
inclusions and natural projections so that the diagram \begin{equation}
\xymatrix @=2pt@! { %We want equal spacing, and the entries close together
&&0\ar[dr]&&0&&0\ar[dr]&&0&&&&0\\
&&&C_{n+1}\ar[ur]\ar[dr]&&&&C_{n-1}\ar[ur]\ar[dr]&&&&C_{n-3}\ar[ur]\\
\cdots \ar[rr]&&A_{n+2}\ar[rr]\ar[ur]&&A_{n+1}\ar[rr]\ar[dr]&&A_{n}\ar[rr]\ar[ur]&&A_{n-1}\ar[rr]\ar[dr]&&A_{n-2}\ar[ur]\ar[rr]&&\cdots\\
&C_{n-2}\ar[ur]&&&&C_{n}\ar[ur]\ar[dr]&&&&C_{n-2}\ar[ur]\ar[dr]&&&&\\
0\ar[ur]&&&&0\ar[ur]&&0&&0\ar[ur]&&0&&&&
}
\label{eq:knit}
\end{equation} is exact. We emphasize that the crossing maps are either
\emph{inclusions} or \emph{natural projections}. The crossing sequences
are short exact because, e.g., for
\(0 \to C_n \to A_n \to C_{n-1} \to 0\),

\begin{itemize}
\tightlist
\item
  \(C_n\) is the kernel, so injects and passes through \(A_n\) to \(0\)
  in \(C_{n-1}\),
\item
  \(A_n\) surjects onto \(C_{n-1}\) and passes to \(0\),
\item
  any element in \(A_n\) that passes to \(0\) in \(C_{n-1}\) must be in
  the image of \(C_n\) in \(A_n\), because \(A_n \to C_{n-1}\) is the
  natural projection onto \(A_n / \ker(A_n \to A_{n-1}) = A_n /C_n\).
\end{itemize}

We now construct the desired short exact sequence \eqref{eq:wts}.
Placing the long exact sequence of homology groups \eqref{eq:les} into
the position of the long exact sequence in the cross stitched diagram
\eqref{eq:knit}, one visibly obtains the short exact sequence:
\begin{equation}
\xymatrix{
0 \ar[r] & \coker\big(H_n(X) \xrightarrow{p_*} H_n(X)\big) \ar[r]  & H_n(X; \ZZ_p) \ar[r] & \ker\big( H_{n-1}(X) \xrightarrow{p_*} H_{n-1}(X)\big) \ar[r] & 0
}
\label{eq:sestosplit}
\end{equation}

This sequence \eqref{eq:sestosplit} is isomorphic to the one we wanted
\eqref{eq:wts}, as
\[\coker\big(H_n(X) \xrightarrow{p_*} H_n(X)\big) \cong \ZZ_p \otimes H_n(X)\]
and
\[\ker\big( H_{n-1}(X) \xrightarrow{p_*} H_{n-1}(X)\big) =: \Tor(H_{n-1}(X), \ZZ_p).\]

What remains to be proven is that there is a homomorphism from
\(H_n(X, \ZZ_p)\) down to \(\ZZ_p \otimes H_n(X)\) that (left) inverts
the inclusion of \(\ZZ_p \otimes H_n(X)\) up into \(H_n(X, \ZZ_p)\). My
argument here closely follows {[}2, No. 3.21{]}.

We'll work levelwise for a while in the singular chain complex
\(\Delta_*(X)\). Let \(Z_n\) be the cycle subgroup of \(\Delta_n(X)\)
and \(B_{n-1}\) the boundary subgroup \(\delta \Delta_n(X)\) in
\(\Delta_{n-1}(X)\). Observe that \(Z_n\) includes into \(\Delta_n(X)\)
which projects onto \(B_{n-1}\). So we have the short exact sequence of
free abelian groups \begin{equation}
\xymatrix{
0 \ar[r] & Z_n \ar[r]^f & \Delta_n(X) \ar@/^1pc/ @{-->} [l]^g \ar[r] & B_{n-1} \ar[r] & 0,
}
\end{equation} that splits with \(g \circ f = \id_{Z_n}\). Extend \(g\)
to the quotient \(H_n(X)\) so that the following diagram commutes:
\begin{equation}
\label{eq:extend}
\xymatrix{
\Delta_n(X)\ar[d]^g \ar@{-->}[dr]^{g'} & \\
Z_n \ar[r] & H_n(X)
}\end{equation}

Consider the graded abelian group \(\{H_k(X)\}_{k \in \ZZ}\) in a chain
with trivial differentials
\[\ldots \xrightarrow{0} H_{n+1}(X) \xrightarrow{0} H_n(X) \xrightarrow{0} H_{n-1}(X) \xrightarrow{0} \ldots.\]
Call this chain complex \(H_*(X)\). We induce a chain map \(g_*\) from
\(\Delta_*(X)\) to \(H_*(X)\) with \(g'\):

\begin{minipage}{.5\textwidth}
\begin{equation}
\xymatrix{
\vdots \ar[d]^\partial & \vdots \ar[d]^0\\
\Delta_{n+1}(X) \ar[r]^{g'} \ar[d]^\partial & H_{n+1}(X) \ar[d]^0\\
\Delta_{n}(X) \ar[r]^{g'} \ar[d]^\partial& H_{n}(X) \ar[d]^0\\
\Delta_{n-1}(X) \ar[r]^{g'} \ar[d]^\partial& H_{n-1}(X) \ar[d]^0\\
\vdots & \vdots\\
}\label{eq:trivialchain}\end{equation}
\end{minipage}
\begin{minipage}{.5\textwidth}
\begin{equation}
\xymatrix{
\vdots \ar[d]^{\id\otimes\partial} & \vdots \ar[d]^0\\
\ZZ_p\otimes\Delta_{n+1}(X) \ar[r]^{\id \otimes g'} \ar[d]^{\id \otimes \partial} & \ZZ_p\otimes H_{n+1}(X) \ar[d]^0\\
\ZZ_p\otimes\Delta_{n}(X)   \ar[r]^{\id \otimes g'} \ar[d]^{\id \otimes \partial} & \ZZ_p\otimes H_{n}(X)   \ar[d]^0\\
\ZZ_p\otimes\Delta_{n-1}(X) \ar[r]^{\id \otimes g'} \ar[d]^{\id \otimes \partial} & \ZZ_p\otimes H_{n-1}(X) \ar[d]^0\\
\vdots & \vdots\\
}\end{equation}
\end{minipage}

Observe that
\(\partial(\Delta_{n+1}(X)) = B_n(X) = \ker\left(Z_n \to H_n(X)\right)\)
so that each square in \eqref{eq:trivialchain} commutes. We may then
tensor both chains by \(\ZZ_p\) on the right, obtaining
\[\ZZ_p \otimes \Delta_*(X) \xrightarrow{\id \otimes g_*} \ZZ_p \otimes H_n(X).\]
When we take homology on \(\ZZ_p \otimes \Delta_*(X)\) by definition we
have \(H_*(X, \ZZ_p)\). However, when computing homology in the chain
complex \(\ZZ_p \otimes H_n(X)\), the image of each map in the complex
is \emph{trivial}, so the homology of the complex is
\(\ZZ_p \otimes H_*(X)\).

The map \({\id \otimes g_*}\) of chain complexes induces a homomorphism
\(H_*(\id\otimes g')\) on homology from
\(H_*(X, \ZZ_p) \to \ZZ_p \otimes H_*(X)\). In particular, at the
\(n\)th level,
\begin{equation}\label{eq:finish}H_n(X, \ZZ_p) \xrightarrow{H_*(\id \otimes g')} \ZZ_p \otimes H_n(X).\end{equation}
By construction in \eqref{eq:extend} of \(g'\) as the extension of the
splitting map \(g\) (where \(g \circ f = \id_{Z_n}\)), the induced map
\(H_*(\id \otimes g')\) in \eqref{eq:finish} is a left inverse to the
inclusion of \(\ZZ_p \otimes H_n(X)\) into \(H_n(X, \ZZ_p)\). \qedsymbol

\subsection*{References}
\addcontentsline{toc}{subsection}{References}

\hypertarget{refs}{}
\leavevmode\hypertarget{ref-Bre93}{}%
{[}1{]} G. E. Bredon, \emph{Topology and Geometry}. New York:
Springer-Verlag, 1993.

\leavevmode\hypertarget{ref-Che09}{}%
{[}2{]} J. L. Chen, ``Universal Coefficient Theorem for Homology,''
2009.


\end{document}
